\documentclass[pagesize=pdftex,paper=a4,bibtotoc,12pt]{scrreprt}

%%%%%%%%%%%%%%%%%DATEN%%%%%%%%%%%%%%%%%
%Erstellungsdatum
\date{\today}
%Autor
\author{Martin Kraus, Philipp Schwarz}
\title{Seminar Topologie: Kompakte Räume}

%%%%%%%%%%%%%%%%%pakete%%%%%%%%%%%%%%%%%
%neue Rechtschreibung
\usepackage{ngerman}
 
%\usepackage[T1]{fontenc}
%Umlaute ermölichen utf8 ( latin1)
\usepackage[utf8]{inputenc}
 
% Tabellen
\usepackage{array}
 
% Schriftfarben
\usepackage{color}

%Bilder
\usepackage{float}
\usepackage{floatflt}
\usepackage[pdftex]{graphicx}
\DeclareGraphicsRule{*}{mps}{*}{}
%Math
\usepackage{amsmath}
\usepackage{amssymb}

%Nomenclature
\usepackage[intoc]{nomencl}

%Kopf-/Fußzeilen
\usepackage{fancyhdr}

%Hyperref für Bookmarks im PDF
\usepackage[pdftex, pagebackref]{hyperref}

%%%%%%%%%%%%%%%%%%%%seitencfg%%%%%%%%%%%%%%%%%%%%%%%%%%%%
\pagestyle{fancy}
\fancyhf{}
%Kopfzeile links bzw. innen
\fancyhead[L]{\nouppercase{\leftmark}}
%Kopfzeile rechts bzw. außen
\fancyhead[R]{\thepage}
%Linie open
\renewcommand{\headrulewidth}{0.5pt}
%Fußzeile links bzw. innen
\fancyfoot[L]{\copyright Seminar Topologie: Kompakte Räume - Martin Kraus, Philipp Schwarz}
%Fußzeile rechts bzw. außen
\fancyfoot[R]{\today}
%Linie unten
\renewcommand{\footrulewidth}{0.5pt}

\usepackage[pdftex, pagebackref]{hyperref}
\hypersetup {
pdftitle = {Martin Kraus, Philipp Schwarz},
pdfsubject = {Seminar Topologie: Kompakte Räume}, 
pdfauthor = {Martin Kraus, Philipp Schwarz},
pdfhighlight = {/O},
pdfkeywords = {Kompakte Räume, Satz von Heine-Borel}, colorlinks = {false},
bookmarksnumbered = {true},
citebordercolor = {1 1 1},
linkbordercolor = {1 1 1},
urlbordercolor = {1 1 1},
bookmarksopen = {true},
bookmarksopenlevel = {1}
}
%%%%%%%%%%%%%%%%%%%EIGENE-BEFEHLE%%%%%%%%%%%%%%%%%%%%%%%%
\newcommand{\ncl}[2]{\nomenclature{\textit{#1}}{\textcolor{white}{test}\\#2}}
\newcommand{\glossar}[1]{\textit{#1}\glossary{\textit{#1}}}

%Bilder
\newcommand{\pic}[1]{Abbildung \ref{#1}}
\newcommand{\picp}[1]{Abbildung \ref{#1} Seite \pageref{#1}}
%Tabellen
\newcommand{\tab}[1]{Tabelle \ref{#1}}
\newcommand{\tabp}[1]{Tabelle \ref{#1} Seite \pageref{#1}}
%Darstellung eines Links im Quellenverzeichnis
\newcommand{\urlg}[2]{\url{#1}\\zuletzt abgerufen am #2}
%Verweise
\newcommand{\see}[1]{(siehe Kapitel \ref{#1})}
\newcommand{\seeo}[1]{siehe Kapitel \ref{#1}}
\newcommand{\seea}[1]{(siehe Anhang \ref{#1})}
% Fett und Schief
\newcommand{\textbit}[1]{\textbf{\textit{#1}}}
\newcommand{\beweis}[1]{\textbf{#1}:}
\newcommand{\qed}{\hfill \(\square\)}
% theoremartige Konstrukte
\newtheorem{Def}{Definition}
\newcommand{\defH}[1]{\textbf{#1}}
\newtheorem{Satz}{Satz}
\newtheorem{Lemma}{Lemma}[Satz]
%%%%%%%%%%%%%%%%%%%%NOMENCLATURE/GLOSSAR%%%%%%%%%%%%%%%%%%%
%%%%%%%%%%%%%%%%%%%%NOMENCLATURE/GLOSSAR%%%%%%%%%%%%%%%%%%%
%Nomenclature 
\makenomenclature

%Glossar
\include{glossar}

%%chemie
%\usepackage[version=3]{mhchem}

%own commands
%Merkbox:
%\begin{merkbox}
% text
%nomenclature}
% \newsavebox\TBox
% \newenvironment{merkbox}
%{%\par\noindent
% \begin{lrbox}{\TBox}
% \varwidth{\textwidth-2.5\fboxsep}
% }{\endvarwidth\end{lrbox}%
% \textcolor{orange}{\textbf{Merke:}}\\[1mm]
% \Ovalbox{\usebox\TBox}\par
%}

%markierungen
%\newcommand{\texthigh}[1]{\textcolor{orange}\textbf{#1}}
%\newcommand{\texthead}[1]{\emph{#1:}\hline}
