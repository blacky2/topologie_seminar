\chapter{Grundlegende Definitionen}

\begin{Def}[Topologischer Raum]
	Sei \(X\) eine Menge und \( \mathcal{O} \) eine Menge von Teilmengen von \(X\). Das Paar \((X,\mathcal{O})\) heißt topologischer Raum, wenn folgende Bedingungen erfüllt sind:
	\begin{enumerate}
		\item \( X, \emptyset \in \mathcal{O} \) 
		\item \( U_{i} \in \mathcal{O} \ \forall  i \in \{1, \dots, n\} 
			\Rightarrow \bigcap_{i=1}^n U_{i} \in  \mathcal{O} \)
		\item \( U_{\lambda} \in \mathcal{O} \ \forall \lambda \in \Lambda \Rightarrow \bigcup_{\lambda \in \Lambda} U_{\lambda} \)
	\end{enumerate}
\end{Def}
\textbf{Bemerkung:} Im Folgenden sei \((X,\mathcal{O})\) ein topologischer Raum.
\\
\textbf{Bemerkung:} Mengen \(O \in \mathcal{O} \) heißen offen. Eine Menge \(B \subset X\) heißt abgeschlossen, wenn ihr Komplement \( X \backslash A\) offen ist.

\begin{Def}[Umgebung eines Punktes] 
	Eine Menge \(U \subset X\) heißt \defH{Umgebung eines Punktes} \(p \in X\), wenn es eine offene Menge \(O \subset X\) gibt für die gilt: 
	\(p \in O \subset U\).
\end{Def}

\begin{Def}[Menge aller Umgebungen eines Punktes]
	Sei \(x \in X\) ein Punkt, dann bezeichnet \(U(x)\) die Menge aller Umgebungen von \(x\). 
\end{Def}

\begin{Def}[hausdorffsch]
	Ein topologischer Raum \((X,\mathcal{O})\) heißt \defH{hausdorffsch}, falls \( \forall x,y \in X\) mit \(x\ne y\) gilt:
	\[\exists U \in U(x) \mbox{ und } \exists V \in U(y) : U \cap V = \emptyset \]
\end{Def}

\begin{Def}[Offene Überdeckung]
	Eine Familie \( (U_{\lambda} | \lambda \in \Lambda) \) mit \(U_{\lambda} \in \mathcal{O} \) für alle \( \lambda \in \Lambda \) heißt 	offene Überdeckung von X, wenn gilt:
	\[ \bigcup_{\lambda \in \Lambda } U_{\lambda} = X \]
\end{Def}
\textbf{Bemerkug:} Ist \(X \subset Y\) ein Teilraum und sind die \(U_{\lambda}\) offen in \(Y\), so heißt die Familie \( (U_{\lambda} | \lambda \in \Lambda) \) offene Überdeckung von
\(X\), wenn gilt \(X \subset \bigcup_{\lambda \in \Lambda} U_{\lambda}\).
\\
\textbf{Bemerkug:} \(\Lambda\) ist eine belibige Indexmenge. Ist diese endlicher Mächtigkeit, also \(|\Lambda| < \infty\), so nennt man auch die Überdeckung endlich. 
\\
\textbf{Bemerkug:} Eine Überdeckung \( (U_{\gamma} | \gamma \in \Gamma) \)  heißt Teilüberdeckung von \( (U_{\lambda} | \lambda \in \Lambda \), wenn \(\Gamma \subset \Lambda\). 

\begin{Def}[Offene Kugel]
	Sei \((X,d)\) ein metrischer Raum, \(p \in X\) und \(\epsilon \in \mathbb{R}\). Dann nennt man
	\[ B_{\epsilon}(p) := \{ x \in X | d(p,x) < \epsilon \} \]
	offene Kugel um den Punkt \(p\) mit Radius \(\epsilon\) oder auch \(\epsilon-Umgebung\) von \(p\).
\end{Def}
