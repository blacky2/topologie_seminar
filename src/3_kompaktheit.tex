\chapter{Kompakte Räume}

	\section{Beispiel}
		Man betrachte die Menge \( (0,1) \subset \mathbb{R} \). Auf den ersten Blick, ist nicht klar, warum diese Menge, laut unserer Definition von kompakten Räumen, nicht kompakt ist. 
		Nehme man zum Beispiel die offene Überdeckung \( (-n,n) \subset \mathbb{R} , n \in \mathbb{N} \), diese überdeckt sogar ganz \(  \mathbb{R} \). 
		Und \( (-1,1)  \) überdeckt schon unsere ganze Menge \( (0,1)  \). Also hat diese offene Überdeckung eine endliche offene Überdeckung die ganz \( (0,1)  \) überdeckt.
		Aber laut Definition von kompakten Räumen muss jede offen Überdeckung eine endliche offene Teilüberdeckung besitzen. Betrachtet man nun die offene Überdeckung 
		\( U_{i}=(0,1-\frac{1}{i}), i \in \mathbb{N} \), so ist \(\bigcup_{i \in \mathbb{N}} U{i} = (0,1) \). Wenn \( (0,1)  \) kompakt wäre, 
		so würde es nun eine Teilmenge \(I \subset \mathbb{N} \) geben mit \(\bigcup_{i \in I} = (0,1) \).
		Da \(I\) eine endliche Teilmenge ist, besitzt \(I\) ein Maximum \( m \in \mathbb{N} \). Wählt man \(x \in (1-\frac{1}{m},1) \), dann ist \(x \in (0,1) \) aber nicht 
		in \(\bigcup_{i \in I} U_{i}\). Also war die Annahme falsch, dass jede offen Überdeckung eine endliche Teilüberdeckung enthält. Daher ist der die Menge \((0,1) \) nicht kompakt.
		
		