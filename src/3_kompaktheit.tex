\chapter{Kompakte Räume}
Im folgendenden Kapitel werden kompakte Räume definiert und motiviert. Weiterhin wird 
wird eine Charakterisierung von Kompaktheit gegeben.
Nun endlich zur Definition von Kompakten Räumen.

\begin{Def}[Kompaktheit]
	Ein hausdorffscher topologischer Raum \((X, \mathcal{O})\) heißt \defH{kompakt}, wenn es zu jeder offenen Überdeckung 
	\((U_{\lambda} | \lambda \in \Lambda)\) von \(X\) eine endliche offene Teilüberdeckung gibt, also
	\[ (U_{\lambda} | \lambda \in \Lambda) \mbox{ offene Überdeckung }
     \Rightarrow \exists \Gamma \subset \Lambda : |\Gamma| < \infty \land \bigcup_{\gamma \in \Gamma} U_{\gamma} = X \] 
	gilt.
\end{Def}
Man kann durch Übergang zu Komplementen folgende Charakterisierung hierzu finden:

\begin{Satz}[Charakterisierung Kompaktheit]
	Ein hausdorffscher topologischer Raum \((X, \mathcal{O})\) ist genau dann {\bf kompakt } wenn gilt:
	\begin{align*}
		(A_{\lambda}| \lambda \in \Lambda) \mbox{ Familie } & \mbox{abgeschlossener Mengen in } X \mbox{ mit } 
		\bigcap_{\lambda \in \Lambda} A_{\lambda} = \emptyset \\
		\Rightarrow \exists \Gamma \subset \Lambda : |\Gamma| < \infty & \land \bigcap_{\gamma \in \Gamma} A_{\gamma} = \emptyset
	\end{align*} 
\end{Satz}

Beweis:
\\
\(\Leftarrow\)
\\
Gelte nun für belibige Familien abgeschlossener Mengen \((A_{\lambda}| \lambda \in \Lambda)\) in  \(X\) mit
\(\bigcap_{\lambda \in \Lambda} A_{\lambda} = \emptyset\), dass eine endliche Teilmenge \(\Gamma \subset \Lambda\) existiert mit
\(\bigcap_{\gamma \in \Gamma} A_{\gamma} = \emptyset\) und sei \( (U_{\omega} | \omega \in \Omega) \) offene Überdeckung
von \(X\). So setze \(A_{\omega} = X \backslash U_{\omega} \; \forall \omega \in \Omega\), d.h. \(A_{\omega}\) ist abgeschlossen
und es folgt 
\[\bigcap_{\omega \in \Omega} A_{\omega} = \bigcap_{\omega \in \Omega} X \backslash U_{\omega} =  X \backslash 
\bigcup_{\omega \in \Omega} U_{\omega} = X \backslash X =  \emptyset \]
Nach Voraussetzung aber existiert ein \(\Gamma \subset \Omega\), sodass die obige Gleichung erfüllt bleibt. Demnach ist 
\[ X = X \backslash \emptyset = X \backslash \bigcap_{\gamma \in \Gamma} A_{\gamma} = \bigcup_{\gamma \in \Gamma} X \backslash A_{\gamma} = 
\bigcup_{\gamma \in \Gamma} U_{\gamma} \]
 und es ist zur Überdeckung \( (U_{\omega} | \omega \in \Omega) \) eine endliche Teilüberdeckung gefunden.
 \\
 \(\Rightarrow\)
 \\
 Die Hinrichtung zeigt man analog, indem man Kompaktheit voraussetzt und für eine Familie abgeschlossener Teilmengen zu den 
 offenen Komplementen übergeht.

 Warum ist denn nun Kompaktheit eine solch angenehme Eigenschaft? Nun sie ist so zusagen, wie einleitend bereits erwähnt, 
 ein Bindeglied zwischen dem endlichen und dem unendlichen - zwischen Säcken und Abgründen. Ein Kompakter Raum seinerseits 
 kann durchaus unendlich, ja sogar überabzählbar unendlich viele Elemente enthalten. Überdecken kann man diese aber immer mit
 einem endlichen System aus Teilmengen des Raumes. Somit sind auf Kompakten Räumen Induktionsschlüsse wie folgt möglich:
 

