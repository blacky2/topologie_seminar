\chapter{Kompakte Räume}
Im folgendenden Kapitel werden kompakte Räume definiert und motiviert. Weiterhin wird eine Charakterisierung jener Räume gegeben.
Nun aber endlich zur Definition von Kompakten Räumen.

\begin{Def}[Kompaktheit]
	Ein hausdorffscher Raum \((X, \mathcal{O})\) heißt \defH{kompakt} wenn es zu jeder offenen Überdeckung 
	\((U_{\lambda} | \lambda \in \Lambda)\) von \(X\) eine endliche offene Teilüberdeckung gibt, wenn also gilt:
	\[ (U_{\lambda} | \lambda \in \Lambda) \mbox{ offene Überdeckung }
     \Rightarrow \exists \Gamma \subset \Lambda : |\Gamma| < \infty \land \bigcup_{\gamma \in \Gamma} U_{\gamma} = X \] 
\end{Def}
Man kann durch Übergang zu Komplementen folgende Charakterisierung finden:

\begin{Satz}[Charakterisierung Kompaktheit]
	Ein hausdorffscher Raum \((X, \mathcal{O})\) ist genau dann \textbf{kompakt}, wenn gilt:
	\begin{align*}
		(A_{\lambda}| \lambda \in \Lambda) \mbox{ Familie } & \mbox{in } X \mbox{ abgeschlossener Mengen mit } 
		\bigcap_{\lambda \in \Lambda} A_{\lambda} = \emptyset \\
		\Rightarrow \exists \Gamma \subset \Lambda : |\Gamma| < \infty & \land \bigcap_{\gamma \in \Gamma} A_{\gamma} = \emptyset
	\end{align*} 
\end{Satz}
\beweis{Beweis}
	\\
	\glqq\(\Leftarrow\)\grqq
	\\
	Gelte nun für belibige Familien abgeschlossener Mengen \((A_{\lambda}| \lambda \in \Lambda)\) mit der Eigenschaft
	\(\bigcap_{\lambda \in \Lambda} A_{\lambda} = \emptyset\), dass eine endliche Teilmenge \(\Gamma \subset \Lambda\) existiert mit
	\(\bigcap_{\gamma \in \Gamma} A_{\gamma} = \emptyset\) und sei \( (U_{\omega} | \omega \in \Omega) \) offene Überdeckung
	von \(X\). So setze \(A_{\omega} = X \backslash U_{\omega} \; \forall \omega \in \Omega\), d.h. \(A_{\omega}\) ist abgeschlossen
	und es folgt 
	\[\bigcap_{\omega \in \Omega} A_{\omega} = \bigcap_{\omega \in \Omega} X \backslash U_{\omega} =  X \backslash 
	\bigcup_{\omega \in \Omega} U_{\omega} = X \backslash X =  \emptyset \]
	Nach Voraussetzung aber existiert ein endliches \(\Gamma \subset \Omega\), sodass die obige Gleichung erfüllt bleibt. Demnach ist 
	\[ X = X \backslash \emptyset = X \backslash \bigcap_{\gamma \in \Gamma} A_{\gamma} = \bigcup_{\gamma \in \Gamma} X \backslash A_{\gamma} = 
	\bigcup_{\gamma \in \Gamma} U_{\gamma} \] 
	und es ist zur Überdeckung \( (U_{\omega} | \omega \in \Omega) \) eine endliche Teilüberdeckung gefunden.
	\\
	\glqq\(\Rightarrow\)\grqq
	\\
	Die Hinrichtung zeigt man analog, indem man Kompaktheit voraussetzt und für eine Familie abgeschlossener Teilmengen zu den 
	offenen Komplementen übergeht.
\qed

\section{Motivation}
Warum ist denn Kompaktheit überhaupt eine beachtenswerte Eigenschaft? Nun sie ist sozusagen ein Bindeglied zwischen dem 
Endlichen und dem Unendlichen - zwischen Säcken und Abgründen. Ein Kompakter Raum seinerseits 
kann durchaus unendlich, ja sogar überabzählbar unendlich viele Elemente enthalten. Überdecken kann man diese aber immer mit
einem endlichen System aus offenen Teilmengen des Raumes. Somit ist auf Kompakten Räumen folgende Schlussweise möglich:
\\
\begin{Satz}
	Sei \( (X, \mathcal{O}) \) ein kompakter Raum, \(O \in \mathcal{O}\) und 
	\( \Psi(O) : \mbox{\glqq} O \mbox{ hat die Eigenschaft } \psi \mbox{\grqq} \) eine Aussage, 
	daher die Eigenschaft \(\psi\) trifft entweder auf \(O\) zu oder eben nicht.
	Gilt nun weiter:
	\begin{align}
		U, V \in \mathcal{O} : \Psi(U) \land \Psi(V) &\Rightarrow \Psi(U \cup V) \label{eq:motivation:1}\\
		\forall x \in X \; \exists U_x \in \mathcal{O}, x \in U_x &: \Psi(U_x) \label{eq:motivation:2}
	\end{align}
	Dann ist auch \(\Psi(X)\) wahr, d.h. es gilt die Eigenschaft \(\psi\) für \(X\).
\end{Satz}
%
\beweis{Beweis}
	Es bildet \( (U_x | x \in X) \) wegen \eqref{eq:motivation:2} eine offene Überdeckung von \(X\). Aus der Kompaktheit von \(X\) folgt nun 
	die Existenz einer endlichen Teilüberdeckung, also \(X =  \bigcup_{i=1}^{n} U_{x_i}\) mit geeignetem \( n \in \mathbb{N} \). 
	Mittels vollständiger Induktion zeigt man, dass \(\Psi(X)\) wahr ist.
	\\
	\\
	\underline{Induktions Anfang}: \(\Psi(\bigcup_{i=1}^{1} U_{x_i}) = \Psi(U_{x_1}) \) ist wahr nach \eqref{eq:motivation:2}.
	\\
	\underline{Induktions Voraussetzung}: Sei nun \( \Psi(\bigcup_{i=1}^{k} U_{x_i}) \) erfüllt.
	\\
	\underline{Induktions Schritt}: Nach (I.V.) und \eqref{eq:motivation:2} ist \(\Psi(\bigcup_{i=1}^{k} U_{x_i}) \land \Psi(U_{x_{k+1}})\) wahr.
	Nun Folgt aus \eqref{eq:motivation:1}, dass \( \Psi(\bigcup_{i=1}^{k} U_{x_i} \cup U_{x_{k+1}}) = \Psi( \bigcup_{i=1}^{k+1} U_{x_i} ) \) erfüllt ist.
	\\
	\\
	Also ist \(\Psi(X) =  \Psi(\bigcup_{i=1}^{n} U_{x_i})\) wahr.
\qed
\\
\\
Man beachte das hier Vollständige Induktion nur anwendbar ist, da eine endliche offene Teilüberdeckung von \( (U_x | x \in X) \) existiert bzw. \(X\) kompakt ist.
Ist zum Beispiel die Funktion \(f : X \to \mathbb{R}\) lokal beschränkt, d.h. zu jedem \(x \in X\) existiert eine Umgebung \(U_x\), mit \(f(U_x)\) ist beschränkt, 
dann ist schon \(f\) beschränkt (Für \(U,V \in \mathcal{O}\) ist \(f(U \cup V)\) durch das Minimum der unteren und das Maximum der oberen 
Schranken von \(f(V), f(U)\) beschränkt). Setzt man \(f(x):=\frac{1}{x}\) und \(X:=(0,1)\), so ist \(f\) lokal beschränkt.
Für \(f(X)\) jedoch lässt sich keine obere Schranke angeben, weil \((0,1)\) nunmal nicht kompakt ist.


\section{Beispiel}
Man betrachte die Menge \( (0,1) \subset \mathbb{R} \).\\ Auf den ersten Blick, ist nicht klar, warum diese Menge, laut unserer Definition von kompakten Räumen, nicht kompakt ist. 
Nehme man zum Beispiel die offene Überdeckung \( S_{n} := (-n,n) \subset \mathbb{R} , n \in \mathbb{N} \), diese überdeckt sogar ganz \(  \mathbb{R} \). 
Und \( S_{1} = (-1,1)  \) überdeckt schon unsere ganze Menge \( (0,1) \). \\
Also hat diese offene Überdeckung eine endliche offene Überdeckung die ganz \( (0,1)\) überdeckt.
Aber laut Definition von kompakten Räumen muss jede offene Überdeckung eine endliche offene Teilüberdeckung besitzen. Wir haben nun aber nur eine offene Überdeckung betrachtet. Betrachtet man nun nämlich die offene Überdeckung \( U_{i}=(0,1-\frac{1}{i}), i \in \mathbb{N} \), so ist \(\bigcup_{i \in \mathbb{N}} U{i} = (0,1) \). Wenn \( (0,1)  \) kompakt wäre, 
so würde es nun eine Teilmenge \(I \subset \mathbb{N} \) geben mit \(\bigcup_{i \in I} = (0,1) \).
Da \(I\) eine endliche Teilmenge ist, besitzt \(I\) ein Maximum \( m \in \mathbb{N} \). Wählt man \(x \in (1-\frac{1}{m},1) \), dann ist \(x \in (0,1) \) aber nicht in 
\(\bigcup_{i \in I} U_{i}\). Also war die Annahme falsch, dass jede offene Überdeckung eine endliche Teilüberdeckung enthält. Daher ist der die Menge \((0,1) \) nicht kompakt.
		
