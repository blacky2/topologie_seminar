\chapter{Vorwort}

%\being{quotation}
F. Bernstein übermittelt noch die folgenden Bemerkungen:
\glqq \dots Von besonderem Interesse dürfte folgende Episode sein: Dedekind äußerte, hinsichtlich 
des Begriffes der Menge: er stelle sich eine Menge vor wie einen geschlossenen
Sack, der ganz bestimmte Dinge enthalte, die man aber nicht sähe, und von denen man
nichts wisse, außer daß sie vorhanden und bestimmt seien. Einige Zeit später gab Cantor
seine Vorstellung einer Menge zu erkennen: Er richtete seine kolossale Figur hoch auf,
beschrieb mit erhobenem Arm eine großartige Geste und sagte mit einem ins Unbe-
stimmte gerichteten Blick: \glqq Eine Menge stelle ich mir vor wie einen Abgrund.\grqq \grqq 
%\end{quotation}
\\
\\

Auf die Frage, welche Überlegungen nun Cantor - den Schöpfer der Mengenlehre - 
zu solch einer Aussage bewogen hatten, kann sicherlich nur eine spekulative Antwort folgen.
Nichtsdestotrotz soll hier im Folgenden kurz ein Versuch gewagt werden. Ein Versuch der lediglich
eigener Anschauung entspringt und so logischer Strenge weder genügen kann noch will.

Man stelle sich, ausgehend von Dedekind, einen Sack vor der bestimmte Dinge enthält.
Dieser ist fest umrandet und somit gut greifbar für Verstand und Hand. Nun wird man
aber beobachten, dass beim befüllen eines solchen Gebindes mit nur endlich vielen voluminösen Dingen 
die Kapazität des selbigen überschritten werden kann - das ewige \"Müllsackdilemma\".
Was bei endlichen Mengen noch funktionieren kann, ist bei undendlichen dann aber zum scheitern verurteilt.

Da ist es doch gleich viel besser einen Abgrund zur Hand zu haben, am Besten unendlich tief. So wäre
das Problem des befüllens gelöst, jedoch sollte diese Vorstellung, der unendlichen Tiefe wegen, für unseren
Verstand nicht mehr all zu gut zugänglich sein.

Die vorliegende Seminararbeit befasst sich mit dem Begriff der kompakten Räume aus der mengentheoretischen
Topologie. Ein Begriff der eine Brücke schlägt zwischen dem für uns greifbaren endlichen und dem
unendlichen - also dem Sack und dem Abgrund wenn man so will.

Hauptziel dieser Seminararbeit ist es den Satz von Heine-Borel zu beweisen. Also das eine Teilmenge des
\(\mathbb{R}^n\) genau dann ein kompakter Unterraum ist, wenn sie beschränkt und abgeschlossen ist.
Des Weiteren werden einige Veralgemeinerungen von bereits aus der Analysis bekannten Sätzen abfallen.
Hier sei der Satz von Bolzano-Weiherstraß als Beispiel erwähnt.

%
% - 1 kapitel allgemiene Definitonen (siehe kraus-schwarz
% - 2 kapitel Komplakte Räume
%     - allgemeine eigenschaften (übergang zum Abgeschlsossenen)
%     - motivation
% - 3 Sätze zu Kompakten Räumen
%     - maximum / minimum (philipp)
%     - weiher straß (martin)
%     - Heine Borell
%

%\begin{quote}
%Blablabalbal
%\end{quote}
%TODO ... dass mal weg machen!!!
%Bibliographie Test:\\
%\cite{book:hardware:pc}
%Die Hardware seite:
%-> Die Anfänge (Röhre über Transistor bis zum Integrierten Schaltkreis) (kurz)
%-> DRAM - SRAM Speichertüpen / Funkionsweise
%-> Fehlerkorrektur / Parität (evtl.)
%-> Optimierungen / zusammenspiel mit prozessor 
%	-> DMA
%	-> Multiprozessor / Core (NUMA)
%	-> L1 / L2 Cache (Prozessor Cache)
%-> MRAM - Die Zukunft

%Die Software seite:
%-> Festplatte als Arbeisspeicher Swap (Pageing)
%-> VRAM (Virtueller RAM) 
%	-> Addresierung 
%	-> Speicherschutz
