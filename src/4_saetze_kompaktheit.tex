\chapter{Sätze über kompakte Räume}

\begin{Satz}
Sei \( (X, \mathcal{O}_x) \) ein kompakter Raum und \((Y, \mathcal{O}_y)\) hausdorfscher Raum, sowie \(f: X \mapsto Y\) stetig.
Dann ist auch das Bild \( f(X) \) kompakt.
\end{Satz}
\textbf{Beweis:}
\\
Sei \( (U_{\lambda} | \lambda \in \Lambda) \) eine offene Überdeckung von \(f(X)\). Es ist zu zeigen,
dass eine endliche Teilüberdeckung existiert. 
\[ f(X) = \bigcup_{\lambda \in \Lambda} U_{\lambda} \Rightarrow X = 
   f^{-1}(\bigcup_{\lambda \in \Lambda} U_{\lambda}) = 
	 \bigcup_{\lambda \in \Lambda} f^{-1}(U_{\lambda}) \]
Da \(f\) stetig ist, sind die Urbilder \( f^{-1}(U_{\lambda}) \) der offenen Mengen \(U_{\lambda}\) wieder offen für 
alle \(\lambda \in \Lambda\). Somit ist \( ( f^{-1}(U_{\lambda}) | \lambda \in \Lambda ) \) eine offene Überdeckung
von \(X\). Zu dieser gibt es aufgrund der Kompaktheit von \(X\) ein endliches \( \Gamma \subset \Lambda \), mit 
der Eigenschaft \( X = \bigcup_{\gamma \in \Gamma} f^{-1}(U_{\gamma}) \). Hieraus folgt:
\[ f(X) = f(\bigcup_{\gamma \in \Gamma} f^{-1}(U_{\gamma})) = 
   \bigcup_{\gamma \in \Gamma} f(f^{-1}(U_{\gamma})) = 
   \bigcup_{\gamma \in \Gamma} U_{\gamma} \]

\begin{Satz}
	Ein kompakter Teilraum \(K\) eines metrischen Raumes \( (X, d) \) ist beschränkt, daher es gilt:
	\[ \exists p \in K, n \in \mathbb{N} : K \subset B_n(p) := \{ x \in X | d(x,p) < n \} \]
\end{Satz}
%TODO Teilraum. .. 
\textbf{Beweis:}
Das System \( (B_{n}(p) | n \in \mathbb{N}) \) der offenen Kugeln vom Radius \(n \in \mathbb{N}\) überdeckt 
den gesamten metrischen Raum \(X\), da man für jedes \(x \in X\) ein \(n \in \mathbb{N}\) findet mit \(d(p,x) < n\)
also mit \(x \in B_{n}(p)\).
So gilt \(K \subset X = \bigcup_{n \in \mathbb{N}} B_{n}(p)\). 
Also wird insbesondere auch \(K\) überdeckt. 
Da \(K\) kompakt ist gibt es ein \(m \in \mathbb{N} : K \subset \bigcup_{i=1}^{m} B_{i}(p) \). Hier ist
mit eingegangen, dass jede endliche Teilmenge von \(\mathbb{N}\) ein Maximum besitzt.
Weiterhin gilt \( B_0(p) \subset B_1(p) \subset \dots \subset B_m(p) \) und es folgt \(K \subset B_m(p)\).

\begin{Satz}
	Sei \(I = [ a , b ] \subset \mathbb{R}\) ein Intervall mit \(a<b\) und \(a,b \in \mathbb{R}\). Dann 
	ist \(I\) kompakt.
\end{Satz}
\textbf{Beweis:}

